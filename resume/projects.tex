\cvsection{Projects}

\begin{cventries}
  \cventry
    {} % Empty position
    {NiN mapping} % Project
    {} % Empty location
    {Jul. 2020} % Empty date
    {
      \begin{cvitems} % Description(s) bullet points
        \item {Took part in \textit{Natur i Norge} mapping study in Karasjok (Norway) as a part of GEco group under NHM - UIO.}
		\item {The project was based around the mapping of a study area by multiple mappers, and subsequent analysis of mapping inconsistencies.}
      \end{cvitems}
    }

  \cventry
    {} % Empty position
    {GIS data scrubbing and analysis} % Project
    {} % Empty location
    {Jan. 2018 - PRESENT} % Empty date
    {
      \begin{cvitems} % Description(s) bullet points
        \item {Django based solution for GIS data gathering from legacy sources}
		\item {Personal project that gathers forestry data in Slovakia from old public servers where the only public option to access the data is through \mbox{Silverlight} web app. This solution gathers the available data while keeping track of sparse changes in time. Metadata for individual polygons are gathered from .NET server, decoded from binary XML files and stored in organized \mbox{PostGis} database.}
      \end{cvitems}
    }

  \cventry
    {} % Empty position
    {Android app for camping} % Project
    {} % Empty location
    {Jul. 2017 - Aug. 2017} % Empty date
    {
      \begin{cvitems} % Description(s) bullet points
      	\item {App designed to support exploration of natural surroundings for tourists in Garvikstrondi camping (Seljord, Norway). \underline{\href{https://play.google.com/store/apps/details?id=sk.malobysa.www.garvikstronditur&hl=en}{Google Play store}}}
      	\item {Offline-capable map including tracks of hikes in the area with detailed descriptions}
      \end{cvitems}
    }

  \cventry
    {} % Empty position
    {Accelerometric data analysis} % Project
    {} % Empty location
    {Aug. 2015 - Jul. 2017} % Empty date
    {
      \begin{cvitems} % Description(s) bullet points
      	\item {Unsupervised machine learning project for analysis of accelerometric data from brown bear (\textit{Ursus arctos}) to identify distinct animal behaviors. Project used Keras library for deep neural networks}
      \end{cvitems}
    }
\end{cventries}
